\documentclass{article}

\usepackage{mathtools}
\usepackage[margin=0.5in]{geometry}
\usepackage{graphicx}
\usepackage{appendix}

\title{Neural Networks Project}
\author{Alfred Arsenault,Brenden Case,David Ashenden,Pranay Vishwanath}
\begin{document}
\maketitle
%Documentation⎯The final report should include the following elements:

\section{Abstract} 
It is known that when investigating the microbiological flora in a given environmental niche, microbial culture is a poor approach due to the inability of culture to sustain all environmental microbes \cite{whymeta}. Hence, to investigate certain niches, we require the use of metagenomics- direct study of genetic material from the niche without culturing. One major computational challenge that this necessitates is metagenomic classification- the classification of genetic data depending on source \cite{current}. In this project, we address the problem of sequence classification using Convolutional Neural Networks (CNNs). Herein, we attempt to classify DNA as human or HIV, from mixed DNA strings that we generated programmatically as a proof-of-concept. Further, we examine the effects of filter choice on classification accuracy and speed.\\
%describes the goals and purpose of your network in 1 page.
\pagebreak

\section{Problem statement}
% A description of your problem, the network design and how it addresses
% the problem in 1-2 pages or less.
\smallbreak

\section{Performance}
%A description of computational performance in 2 pages or less.
\smallbreak

\section{Improvements}
% An analysis of the performance and how it might be improved in the
% future. This analysis should include appropriate statistical methodologies
% in 2 pages or less.
% Important consideration - working with "ideal"/"noise-free" signal;
% real world usage will need to account for noise and for unclear/ambiguous nucleotides
\smallbreak

\section{Conclusion}
\smallbreak

\section{Discussion}
% A summary and conclusion addressing the overall performance of the
% network and what might be done to improve it in the future. This could
% take the form of conjectures, references to other works, etc. and should be
% 1 page.

\begin{thebibliography}{99}
\bibitem{whymeta}
Hugenholtz P., Goebel B.M., and Pace N.R. (1998). Impact of Culture-Independent Studies on the Emerging Phylogenetic View of Bacterial Diversity. \textit{Journal of Bacteriology, 180}(18), 4765–4774. 

\bibitem{current}
Breitwieser F., Lu J., Salzberg S. (2017). A review of methods and databases for metagenomic classification and assembly. \textit{Briefings in Bioinformatics}. DOI: https://doi-org.ezp.welch.jhmi.edu/10.1093/bib/bbx120

\end{thebibliography}

\smallbreak
\appendix
\section{Code Appendix}

\end{document}
